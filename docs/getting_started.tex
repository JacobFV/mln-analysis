\hypertarget{getting-started}{%
\section{Getting Started}\label{getting-started}}

\hypertarget{requirements}{%
\subsection{Requirements}\label{requirements}}

\begin{itemize}
\tightlist
\item
  \textbf{Install node.js and npm}. This server was developed and tested
  using the node.js v16.13.1 and npm 8.1.2 writing. Follow the
  directions at \href{https://nodejs.org/en/download/}{this link} to
  install node.js and npm.
\end{itemize}

\hypertarget{installation}{%
\subsection{Installation}\label{installation}}

\begin{enumerate}
\def\labelenumi{\arabic{enumi}.}
\item
  Clone this repository and install the dependencies:

\begin{Shaded}
\begin{Highlighting}[]
\FunctionTok{git}\NormalTok{ clone https://github.com/JacobFV/mln-dashboard.git}
\BuiltInTok{cd}\NormalTok{  mln-dashboard}
\ExtensionTok{npm}\NormalTok{ install}
\end{Highlighting}
\end{Shaded}
\item
  Next, rename \texttt{.env.example} to \texttt{.env}. Then follow the
  directions included as comments in \texttt{.env} to set up your
  environment. Most of these variables are required for the application
  to run. If you do not want to use one of the auth providers, you will
  need to comment it out in the
  \texttt{/src/pages/api/auth/{[}...nextauth{]}.js} file.
\item
  Initialize the database by 1) changing
  \texttt{"isolatedModules":\ true} to
  \texttt{"isolatedModules":\ false} in \texttt{tsconfig.json} and then
  2) running \texttt{npx\ prisma\ migrate\ dev\ -\/-name\ init}. Make
  sure you see the console display \texttt{seeding\ database} so you
  know that the database is working. You can inspect the database by
  running \texttt{npx\ prisma\ studio}.
\item
  You should now be able to run the application by running
  \texttt{npm\ run\ dev}. Click on the URL shown in the terminal to get
  started!
\end{enumerate}

\hypertarget{getting-help}{%
\subsection{Getting Help}\label{getting-help}}

\begin{itemize}
\item
  Before you start programming or seriously reading the code, I
  recommend setting aside an hour to read
  \href{https://nextjs.org/learn/foundations/from-javascript-to-react}{this
  guide on react},
  \href{https://nextjs.org/learn/foundations/from-react-to-nextjs}{this
  guide on next.js}, and
  \href{https://nextjs.org/learn/foundations/how-nextjs-works}{this
  guide on advanced next.js features}.
\item
  Make sure you can get the server running before you make any changes
  to the code.
\item
  Prisma Studio is your friend. Run \texttt{npx\ prisma\ studio} to see
  what the database looks like.
\item
  If you are using VS Code, you can enjoy break-point debugging within
  your IDE by debugging with either the
  \texttt{Next.js:\ debug\ server-side} or
  \texttt{Next.js:\ debug\ full\ stack} configurations.
\item
  Also, if you are using vscode, you will notice the following
  extensions are recommended: ``NextJS Developer Extensions Pack'',
  ``Prisma'', and ``Rainbow Theme''. Combined, I find they dramatically
  improve the development experience.
\item
  \href{https://github.com/features/copilot/signup}{Sign up} for
  \href{https://copilot.github.com/}{Github CoPilot} if you haven't
  already. My experience is that copilot makes it easy to jump into new
  languages and frameworks.
\item
  If you are stuck, please contact me at
  \texttt{jacobfv123\ {[}at{]}\ gmail\ {[}dot{]}\ com}.
\end{itemize}
